% This must be in the first 5 lines to tell arXiv to use pdfLaTeX, which is strongly recommended.
\pdfoutput=1
% In particular, the hyperref package requires pdfLaTeX in order to break URLs across lines.

\documentclass[a4paper, 11pt]{article}
% \documentclass[sigconf,anonymous]{acmart}

\bibliographystyle{unsrtnat}
\usepackage[numbers,sort&compress]{natbib}
% Remove the "review" option to generate the final version.
\usepackage[]{acl}

% Standard package includes
\usepackage{times}
\usepackage{latexsym}

% For proper rendering and hyphenation of words containing Latin characters (including in bib files)
\usepackage[T1]{fontenc}
% For Vietnamese characters
% \usepackage[T5]{fontenc}
% See https://www.latex-project.org/help/documentation/encguide.pdf for other character sets

% This assumes your files are encoded as UTF8
\usepackage[utf8]{inputenc}

% \pagenumbering{roman}

% \usepackage[utf8]{inputenc}
 
% \pagenumbering{alph}
% This is not strictly necessary, and may be commented out,
% but it will improve the layout of the manuscript,
% and will typically save some space.
\usepackage{microtype}

% \usepackage{hyperref}
\usepackage{xurl}
\PassOptionsToPackage{hyphens}{url}\usepackage{hyperref}
\usepackage{filecontents}
\usepackage{graphicx}
% \usepackage{graphics}
\usepackage{array}

% If the title and author information does not fit in the area allocated, uncomment the following
%
%\setlength\titlebox{<dim>}
%
% and set <dim> to something 5cm or larger.

\title{Pre-trained Language Model: Survey}

% Author information can be set in various styles:
% For several authors from the same institution:
% \author{Author 1 \and ... \and Author n \\
%         Address line \\ ... \\ Address line}
% if the names do not fit well on one line use
%         Author 1 \\ {\bf Author 2} \\ ... \\ {\bf Author n} \\
% For authors from different institutions:
% \author{Author 1 \\ Address line \\  ... \\ Address line
%         \And  ... \And
%         Author n \\ Address line \\ ... \\ Address line}
% To start a seperate ``row'' of authors use \AND, as in
% \author{Author 1 \\ Address line \\  ... \\ Address line
%         \AND
%         Author 2 \\ Address line \\ ... \\ Address line \And
%         Author 3 \\ Address line \\ ... \\ Address line}

%for single author (just remove % characters)
\author{
{\rm Wong Tsz Ho}\\
Student ID: 20725187\\
thwongbi@connect.ust.hk
% copy the following lines to add more authors
% \and
% {\rm Name}\\
%Name Institution
} % end author


% \pagestyle{headings}
% \rfoot{Page \thepage}

\begin{document}
\maketitle
\thispagestyle{plain} % for page number
\pagestyle{plain} % for page number
\begin{abstract}
The survey aims to examine pre-trained language models. We will cover everything from defining a model to the evolution of it. Various popular pretrained language models with techniques and methodologies will be covered, including BERT, GPT, T5, OPT, and LaMDA. A brief review of recent developments in benchmarking will also be provided. Several Python libraries will be presented, including NLTK, SpaCy, HuggingFace, TensorFlow, and PyTorch. We will conclude our survey with a discussion on the application of the pre-trained language models.
\end{abstract}

\section{Introduction}

Among the hottest topics in NLP, pre-trained language models have gained popularity. The model opened up the possibility of a generic and universal language model that solves downstream language problems and functions as a knowledge base. This is a significant milestone in the creation of Strong AI. \\
\\
In this survey, I will discuss the evolution of the pre-trained model from the past until now, and introduce the most popular pre-trained language model using benchmarking. Following that, I will discuss the existing tools for applying those language models, and the survey will conclude with some applications of those pre-trained language models. By putting this information into context, audiences will be able to gain a better understanding of the current development of pre-trained language models, allowing them to step into this exciting area with ease.

\section{Language Model}

A language model is just a probability distribution of the sequence of words. Simply saying, it is the probability of saying that the word sequence is making sense or not. In most cases, the input of a language model would be a sentence, and the language model will determine whether or not the sentence is meaningful.\\
\\
The reason why a language model is so important to neutral language processing is that the language model gives meaning to the mathematical model and the machine learning model can now take the language sense into account when they made the decision.\\
\\
The recent development of language models would be on masked language models. Instead of predicting the next word, masked language models can handle masked words in the middle of the word sequence. BERT would be one of the masked language models.
\section{Language Model Taxonomy}

\subsection{Statistical Language Models}

\subsubsection{N-gram Language Models}
Way before the rise of the neural network, scientists find some ways to represent a language with n-gram. N-gram is a moving N window of a word sequence and we use the distribution of the combination of n-grams to predict the next word sequence. \\

A unigram language model does not take the sequence of the word into its model and only generates the word sequence using the distribution over the vocabulary. While bigram model takes two words into account and calculates the probability distribution over the corpus. Using a high value in N can improve the meaning of the generated word sequence but comes at the cost of computation and storage problems.

\subsection{Neural Network Language Model}

Recurrent Neural Network like LSTM\cite{LSTM} introduces sequencing with the input element where language by nature has order. RNN also allows the model weight to be constant with respect to any length of the input because the model ingests input one by one.  \\
\\
In 2003, Bengio\cite{10.5555/944919.944966} proposed that word embedding can be learned using a neural network. Since then the research on neural Network Language Model boomed. The concept of using words as vectors create many possibilities for treating words. Pretrained word2vec\cite{mikolov2013efficient} model, from Google, then appears in the market and caused a huge impact on the NLP community. The benefit of expressing the words as vectors is that you can treat those words as numbers and are able to do mathematical operations like addition, subtraction, etc. Something like, "girl"+"women"-"men" = "boy" would make sense in vector space. Word2Vec is not the only pre-trained word embedding, GloVe\cite{pennington2014glove}, from Stanford University, is also the other popular word representation. This difference between the two models is that Glove would consider the co-occurrences information. FastText\cite{FastText}, from Facebook, further takes unknown words from generalization. 

\section{Pre-trained Language Model}

A pre-trained language model nowadays usually is trained on a large corpus where the model can represent the universal language. Using those pre-trained language model can save us times and effort to work on everything from scratch. Several hundred models parameters have been added to deep learning models since the advent of deep learning. \\
\\
For full parameter training and to prevent overfitting, a much larger dataset is required. The high annotation costs for NLP tasks, especially for the semantically and syntax-related tasks, make constructing large-scale labeled datasets a challenge for most NLP tasks. Pre-trained language model is indeed helping the evolution of humanity to accelerate faster than ever before. To see how far we have come, I will recount a few of the pre-trained language models that we have developed.

\subsection{BERT}

In 2018, BERT \cite{DBLP:journals/corr/abs-1810-04805}or Bidirectional Encoder Representation from Transformers was born in Google. It is a Transformer Language Model. It consists of self-attention heads and multiple encoder layers. With its bidirectional nature, it does not encode the same word into one vector, instead, it will consider the semantics of the sentence. \\

\begin{figure}[h]
    \centering
    \includegraphics[scale=0.55]{images/BERT.png}
    \caption{BERT\cite{DBLP:journals/corr/abs-1810-04805}: Input Representation}
    \label{fig:mesh1}
\end{figure}

BERT model is pre-trained together with the masked language model and next sentence prediction. This allows the combined loss function would be as low as possible. As mentioned earlier in the survey, a masked language model is the language model that will randomly mask out the words as an input to the model. For BERT, 15\% of the words are masked and the masked word is replaced by a [MASK] token. Next sentence prediction from BERT will take two sentences as input and learn whether or not they are the next sentence. Combining these two training will result in a lower loss of the entire model. BERT model is available on \href{https://huggingface.co/docs/transformers/model_doc/bert}{HuggingFace}.

\subsection{GPT}

GPT stands for Generative Pre-trained Transformer, and the latest version is GPT-3\cite{DBLP:journals/corr/abs-2005-14165}. It is developed by OpenAI. GPT-3 trained on 175B parameters with open-source dataset named 'Common Crawl' which has around 45TB of text. GPT-3 is not open source like GPT\cite{radford2018improving} and GPT-2\cite{radford2019language} and hence you cannot find GPT-3 online. Pre-trained GPT-1/GPT-2 models are in \href{https://huggingface.co/docs/transformers/model_doc/gpt2}{HuggingFace} and you can directly implement the model using the Python package. There is one benefit of using GPT-3 is that it does not require developers to fine-tune to perform tasks. 

\subsection{T5}

Text-To-Text Transfer Transformer\cite{2020t5}, also known as  T5, suggest that every task we would like to do is an question and answer pairs. Tasks like translation, classification, chat bot, would input as text to the T5 model and the T5 model will generate the target text. For example, to ask T5 to translate "This is awesome." to Chinese, your input to the T5 model would be "translate English to Chinese: This is awesome.". You can find the T5 model on \href{https://huggingface.co/docs/transformers/model_doc/t5}{HuggingFace}.


\subsection{OPT}

Meta AI proposed the OPT model in the paper Open Pre-trained Transformer Language Models\cite{OPT}. OPT is a family of large, open-sourced causal language models that perform similarly to GPT3. A difference between GPT2 and OPT is that OPT adds the EOS token </s> at the beginning of every prompt. Same as other transformers model, you can find the OPT model on \href{https://huggingface.co/docs/transformers/model_doc/opt}{HuggingFace}.

\subsection{LaMDA}

LaMDA\cite{thoppilan2022lamda}, Language Model for Dialogue Applications, is a generative language model by Google. It is also built on Transform. The special thing about this model is that it is trained on the dialog rather than the web text or wiki page like in GPT. It gives that LaMDA model is for Google to improve their Google assistant. And they just announced that they had released the next version LaMDA in Google I/O 2022, which is built on a new Pathways\cite{chowdhery2022palm} system which allows the model to scale to 540B parameters.

\section{Benchmarking a Language Model}

Benchmarking is important for a researcher to evaluate how well their models are. GLUE\cite{wang2018multitask}, General Language Understanding Evaluation benchmark, is one of the popular benchmarking metrics. It provides tools for evaluating and analyzing the performance of models of natural language understanding across a range of existing tasks proposed as part of the GLUE. With the rapid development of the NLP models, most of the models can score really high in their dataset and benchmark, and hence, two years after the launch of GLUE, they launch the successor of this benchmark, which is called SuperGLUE\cite{wang2020superglue}.  
  
\section{Language Model with Python}

\subsection{Language Model library}

\subsubsection{NLTK}

NLTK\cite{NLTKBook} is the Natural Language Toolkit in Python. It is free and open-source. It provides useful utility to work on word with Python. With NLTK, you can tokenize and tag text with ease and hence you can identify named entities. It embedded datasets where you can fine-tune your custom model with NLTK. NLTK comes with a package called NLTK.LM where it currently only supports n-gram language models. NLTK does not provide a neural network language model for more advanced use.\\

\subsubsection{SpaCy}

SpaCy\cite{spacy2}, when compare to NLTK, can process NLP task faster. It provides pre-trained models and a pipeline for building your NLP application. There are 4 pre-trained pipelines for English, \textit{en\_core\_web\_sm}, \textit{en\_core\_web\_md}, \textit{en\_core\_web\_lg} and \textit{en\_core\_web\_trf}. The name represented here was that it is an English general language model trained from Web text which includes blogs, and news. The last naming represents the size of the model while trf represents that it is a transformer model roberta-base. Spacy package is optimized for CPU running and hence it is suitable for fast inference of model to give an immediate response on the lightweight application. Another special thing spacy provides is that it has a Chinese pre-trained model free to use.\\

In our project, Toxic language detection/ debiasing toxic content, this is the one we use to quickly build the baseline of the task. We tagged the toxic language words using IOB\cite{DBLP:journals/corr/cmp-lg-9505040} format which can give us a fair result on the entity recognition task. After we tagged all of the toxic text from the dataset, we bind the IOB file to the pre-trained SpaCy English model. After hours of fine-tuning the pre-trained model, the pipeline is ready to use. The text will first transform to a vector using tok2vec. Then the tagger files will be loaded and all our toxic text will be bound with the vector and the Named Entity Recognition will identify toxic text from the comments.

\begin{figure}[h]
    \centering
    \includegraphics[scale=0.35]{images/SPACYMODEL.png}
    \caption{SpaCy\cite{spacy2} Pre-trained Pipeline Architecture}
    \label{fig:mesh1}
\end{figure}

\subsubsection{HuggingFace}

HuggingFace\cite{DBLP:journals/corr/abs-1910-03771} rise after the transformer revolution that started in 2016. It is a repository of community-based NLP pre-trained language models. It serves a number of transformer model by tasks that you can easily select the right model for your problem. There are nearly 46k pre-trained models and of course they have the popular BERT and GPT2 .\\
\\
In our project, Toxic language detection/ debiasing toxic content, we utilize HuggingFace provided BERT/DistillBERT model to fine-tune the pre-trained model. Using HuggingFace's pre-trained model saves us time on training the model from scratch. The BERT model has 12 attention layers and all these layers combined has 110 million trainable parameters while 66 million for DistillBERT. Unlike Word2Vec, the tokenizer in BERT/DistillBERT splits tokens into subtokens on less commonly seen word which it helps the model to generate and handle an unseen word.\\

\begin{figure}[h]
    \centering
    \includegraphics[scale=0.35]{images/HuggingFace Logo.png}
    \caption{HuggingFace\cite{DBLP:journals/corr/abs-1910-03771}: Logo and GitHub Stars}
    \label{fig:mesh1}
\end{figure}

\subsubsection{TensorFlow \& PyTorch}

TensorFlow\cite{tensorflow2015-whitepaper} and PyTorch\cite{NEURIPS2019_9015} are two popular generic deep learning library. While you can train your own language model, both of them provide datasets and a pre-trained language model for us to use. TensorFlow has set up a  \href{https://tfhub.dev/s?module-type=text-language-model}{Tensor Hub} for the community to host their models on TensorFlow. Not only can you find Language models there, you can find other pre-trained models in Audio, Video, and Image as well. \href{https://pytorch.org/hub/research-models}{PyTorch Hub} is the place where PyTorch stores its pre-trained model for a developer to use. You can find pre-trained language models like GPT-2, and BERT in PyTorch Hub.

\subsection{Other NLP Python Library}

There are a few more python packages worth mentioning. By Humboldt University Berlin, Flair\cite{akbik2018coling} is a framework designed for developing state-of-the-art natural language processing. It allows us to perform text embedding with simple interfaces on Python. This package is built on top of PyTorch and hence it is fully extensible to other PyTorch applications.\\
\\
Gensim\cite{rehurek2011gensim} provides ready-to-use corpora and models with a streaming algorithm to load and process NLP tasks on the go without loading all data onto your memory. You can easily load and use pre-trained model like word2vec\cite{mikolov2013efficient} and FastText\cite{FastText} for pre-processing or other NLP Task.\\

\subsection{Application of Pre-trained LM}
Apart from what Language Model is built for, predicting the next word and calculating the probability that a sentence makes sense or not, there are a number of interesting tasks that a pre-trained language model can help with. NLP tasks like text categorization, speech recognition, Neural Machine Translation, and information retrieval can all boost the performance with a pre-trained language model. This language model can then be used to perform NLP tasks in different domains.\\
\\
Since a pre-trained language model has a feed with lots of real-world knowledge on the web, it makes sense that sometime the next word prediction can treat it as knowledge for answering the prompt. Like when you ask "Albert Einstein was born in \_\_\_", and the next word that the language model predicts should be the exact date he was born. Because most of the text we feed into the model is some sort of fact in the world and hence we can utilize a pre-trained language model as a knowledge base. Acquiring enough commonsense knowledge, we are one step closer to developing a generic Strong artificial intelligence (AI)\cite{butz2021towards}.\\

Pre-trained Language Model is still an active growing study area and it is so exciting to see how the community and the model evolve over the year hence this survey is research that helps you get into this area and admire how far we have been in this area. 

\bibliographystyle{acl_natbib}
\bibliography{anthology, custom}

\appendix


\end{document}
